\documentclass[11pt,letterpaper]{ctexart}
\textwidth 6.5in
\textheight 9.in
\oddsidemargin 0in
\headheight 0in
\usepackage{graphicx}
\usepackage{fancybox}
\usepackage[utf8]{inputenc} %solucion del problema de los acentos.
\usepackage{epsfig,graphicx}
\usepackage{multicol,pst-plot}
\usepackage{pstricks}
\usepackage{amsmath}
\usepackage{amsfonts}
\usepackage{amssymb}
\usepackage{eucal}
\usepackage[left=2cm,right=2cm,top=2cm,bottom=2cm]{geometry}
\pagestyle{empty}
\DeclareMathOperator{\tr}{Tr}
\newcommand*{\op}[1]{\check{\mathbf#1}}
\newcommand{\bra}[1]{\langle#1 |}
\newcommand{\ket}[1]{| #1 \rangle}
\newcommand{\braket}[2]{\langle#1 | #2 \rangle}
\newcommand{\mean}[1]{\langle#1 \rangle}
\newcommand{\opvec}[1]{\check{\vec#1}}
\renewcommand{\sp}[1]{{\begin{split}#1 \end{split}}}

\usepackage{lipsum}

\usepackage{listings}
\usepackage{xcolor}



% 导入包
\usepackage{hyperref}
% 格式设置
\hypersetup{
    colorlinks=true,
    linkcolor=pink,
    filecolor=magenta,      
    urlcolor=cyan,
    pdftitle={Overleaf Example},
    pdfpagemode=FullScreen,
}


\definecolor{codegreen}{rgb}{0,0.6,0}
\definecolor{codegray}{rgb}{0.5,0.5,0.5}
\definecolor{codepurple}{rgb}{0.58,0,0.82}
\definecolor{backcolour}{rgb}{0.95,0.95,0.92}
\definecolor{backcolourcode}{rgb}{255,255,255}

\lstdefinestyle{mystyle}{
	backgroundcolor=\color{backcolour},   
	commentstyle=\color{codegreen},
	keywordstyle=\color{magenta},
	numberstyle=\tiny\color{codegray},
	stringstyle=\color{codepurple},
	basicstyle=\footnotesize,
	breakatwhitespace=false,         
	breaklines=true,                 
	captionpos=b,                    
	keepspaces=true,                 
	numbers=left,                    
	numbersep=5pt,                  
	showspaces=false,                
	showstringspaces=false,
	showtabs=false,                  
	tabsize=2
}


\lstdefinestyle{C++}{
	backgroundcolor=\color{backcolourcode},
    language        =   C++, 
	tabsize         =   4,
	commentstyle    =   \rmfamily\itshape,  % 注释的风格,斜体
    basicstyle      =   \zihao{-5}\ttfamily,
    numberstyle     =   \zihao{-5}\ttfamily,
    keywordstyle    =   [2] \color{pink},
    stringstyle     =   \color{magenta},
    commentstyle    =   \color{pink}\ttfamily,
    columns         =   fixed,  % 如果不加这一句,字间距就不固定,很丑,必须加
    basewidth       =   0.5em,
	frame           =   lrtb,   % 显示边框
	flexiblecolumns,                % 别问为什么,加上这个
	showspaces      =   false,  % 是否显示空格,显示了有点乱,所以不现实了
}

\lstset{style=mystyle}

\begin{document}
\pagestyle{plain}
\begin{flushleft}


%%%%%%      你可以在这里修改你的个人信息与教师信息       %%%%%%%%%
\textbf{ID 22031212122} \\
\textbf{NAME Xiaoning Shu}\\
\textbf{TEAC Qiulin Huang}\\
\textbf{DATE 20230601}\\

\end{flushleft}

\begin{flushright}\vspace{-35mm}
\includegraphics[height=4.0cm]{logo.png}
\end{flushright}
 
\begin{center}\vspace{-0.1cm}
\textbf{ \large \textit{Matrix Theory}}\\
\end{center}

 
\rule{\linewidth}{0.6mm}
%%%%%%%%%%%%%%%%%%%%%%%%%%%%%%%%%%%%%%%%%%%%%%%%%%%%%%%%%%%%%%%%%%%%%%%%

\bigskip
% \textbf{\large{{REPORT}}} \\

\begin{center}\vspace{-0.1cm}\textbf{ \large Homework Of Matrix Theory Per Chapter}\\
\end{center}

\raggedleft{\href{https://github.com/shuxiaoningAK/Matrix\_Theory\_Xdu\_Homework}{此项目已开源,请点击前往github首页}}

\bigskip
% \textbf{\large{Problems}}

\begin{enumerate}

%%%%%%%%%%%%%%%%%%%%
 \item \textit{HOMEWORK 1  {P25-26  3, 5, 7, 9}}%%% 
%%%%%%%%%%%%%%%%%%%%

\textbf{3 QUE}
\bigskip

	判别下列集合对所指运算是否构成{R}上的线性空间:

	(1) 次数等于$m(m \geqslant 1)$的实系数多项式的集合,对于多项式的加法和数与多项式的乘法;

	(2) 实对称矩阵的集合,对于矩阵的加法和实数与矩阵的乘法;

	(3) 平面上全体向量的集合,对于通常的加法和如下定义的数乘运算 $k \circ x = \vec{0}$.

\textbf{3 ANS}
\bigskip

	(1) 不构成,在加法和在乘法上,两个系数相加或相乘可能会超过$m$,因此不封闭.
	
	(2) 构成,因为实矩阵的加法与乘法都是在相应的加法和乘法结果内,实矩阵相加仍然是实矩阵,实矩阵相乘仍然是实矩阵,构成封闭性。

	(3) 不构成,对于八大法则来说,当$x \neq 0$时, $1 \circ x = 0$时就不满足了。
	
\textbf{5 QUE}
\bigskip
	
求习题3(2)中线性空间的维数与基.

\textbf{5 ANS}
\bigskip

假设$\mathbf{A}$的维度为$n\times n$,则实对称矩阵有对角线上的元素和下三角(或上三角)
的元素共计$\frac{n(n+1)}{2}$个,由于对角线上的元素对称,只需考虑下三角(或上三角)的元素。
因此,实对称矩阵的维度为$\frac{n(n+1)}{2}$。
基的选取可以考虑将下三角(或上三角)的某个元素置为1,其余元素为0


\textbf{7 QUE}
\bigskip

求向量$P_2$中向量$1 + t + t^2$对基: $1, t - 1, (t - 2)(t - 1)$的坐标.

\textbf{7 ANS}
\bigskip

	假设坐标$(a, b, c)$ 即有

	$a * 1 + b * (t - 1) + c * (t - 2)(t - 1) = 1 + t + t^2 $

	解得 
			\[\begin{cases}
				a = 3\\
				b = 4\\
				c = 1\\
		\end{cases}\]

	所以在该基下的坐标为$(3,4,1)^T$.

\textbf{9 QUE}
\bigskip

在 $R^4$ 中有两个基

 \[x_1 = e_1, x_2 = e_2, x_3 = e_3, x_4 = e_4\]

 \[y_1 = (2,1,-1,1), y_2 = (0, 3, 1, 0)\]

 \[y_3 = (5, 3, 2, 1), y_4 = (6, 6, 1, 3) \]

(1)求由前一基改变为后一基的过渡矩阵

(2)求向量$x = (\xi_1, \xi_2, \xi_3, \xi_4)$对后一基的坐标

(3)求对两个基有相同坐标的非零向量

\textbf{9 ANS}
\bigskip

	(1) 根据$y=XC_2$可知

		\[\begin{pmatrix}
			2 & 0 & 5 & 6 \\
			1 & 3 & 3 & 6 \\
			-1 & 1 & 2 & 1 \\
			1 & 0 & 1 & 3 \\
		\end{pmatrix}= 
		\begin{pmatrix}
			1 & 0 & 0 & 0 \\
			0 & 1 & 0 & 0 \\
			0 & 0 & 1 & 0 \\
			0 & 0 & 0 & 1 \\
		\end{pmatrix}C_2\]


		解得\[C_2 = \begin{pmatrix}
			2 & 0 & 5 & 6 \\
			1 & 3 & 3 & 6 \\
			-1 & 1 & 2 & 1 \\
			1 & 0 & 1 & 3 \\
		\end{pmatrix}\]


	(2) $x = yC_1, y = XC_2$ 可知:$C_2^{-1}x = C_1$

	所以x在后一基中的坐标为$C_2^{-1}x$.


	(3) 设 $a = yC, a = XC$
	
	

	解得 $(y - I)C = 0$ 
	
	化简为阶梯型后为:
	\[(y - I) = \begin{pmatrix}
		1 & 0 & 5 & 6 \\
		0 & 1 & 6 & 7 \\
		0 & 0 & 1 & 1 \\
		0 & 0 & 0 & 0 \\
	\end{pmatrix}\]

	{所以最终结果为$a = k(1, 1, 1, -1)^T $\qquad  k为不为0的常数}


%%%%%%%%%%%%%%%%%%%%
\item \textit{HOMEWORK 2 {P25-26 11, 12, 13}}%%% 
%%%%%%%%%%%%%%%%%%%%

\textbf{11 QUE}
\bigskip

求 $R^4$ 的子空间

$V_1 = \{(\xi_1, \xi_2, \xi_3, \xi_4) \mid \xi_1 - \xi_2 + \xi_3 - \xi_4 = 0\}$ 

$V_2 = \{(\xi_1, \xi_2, \xi_3, \xi_4) \mid \xi_1 + \xi_2 + \xi_3 + \xi_4 = 0\}$

的交$V_1 \cap V_2$的基。


\textbf{11 ANS}
\bigskip

上面子空间的矩阵为

\[\begin{pmatrix}
	1 & -1 & 1 & -1 \\
	1 & 1 & 1 & 1 \\
	0 & 0 & 0 & 0 \\
	0 & 0 & 0 & 0 \\
\end{pmatrix}\]

可以进行阶梯最简化,所以秩为2, 所以基为 $y = k_1(0, 1, 0, -1) + k_2(1, 0, -1, 0)$


\textbf{12 QUE}
\bigskip

给定$R^{2\times2} = \{A = (a_{ij})_{2\times2} | a_{ij}= \in R \}$ (数域$R$上的
2阶方阵按通常矩阵的加法与数乘矩阵构成的线性空间)的子集

\[V = \{A = (a_{ij})_{2\times2} | a_{ij} \in R \quad and \quad  a_{11} + a_{22} = 0 \} \]

(1)证明$V$是$R^{2\times2}$的子空间

(2)求$V$的维数和一个基



\textbf{12 ANS}
\bigskip

(1) 要证明 $V$ 是 $R^{2\times2}$ 的子空间,我们需要验证以下三个条件:

(i) 零向量属于 $V$:

满足 $a_{11} + a_{22} = 0$ 的零矩阵:零矩阵满足条件,属于 $V$。

(ii) 加法封闭性:

假设 $A = \begin{bmatrix} a_{11} & a_{12} \\ a_{21} & a_{22} \end{bmatrix}$ 和 $B = \begin{bmatrix} b_{11} & b_{12} \\ b_{21} & b_{22} \end{bmatrix}$ 是 $V$ 中的任意两个矩阵,即 $a_{11} + a_{22} = 0$ 和 $b_{11} + b_{22} = 0$。

$(A + B)_{11} + (A + B)_{22} = 0$。

由于 $a_{11} + a_{22} = 0$ 和 $b_{11} + b_{22} = 0$,我们可以得到:

\[(A + B)_{11} + (A + B)_{22} = 0 + 0 = 0\]

因此,$A + B$ 也满足条件,属于 $V$。

(iii) 数乘封闭性:

假设 $A = \begin{bmatrix} a_{11} & a_{12} \\ a_{21} & a_{22} \end{bmatrix}$ 是 $V$ 中的矩阵,即 $a_{11} + a_{22} = 0$。$k$ 是任意的实数。

证明 $(kA)_{11} + (kA)_{22} = 0$。

\[(kA)_{11} + (kA)_{22} = k(a_{11}) + k(a_{22})\]

由于 $a_{11} + a_{22} = 0$

\[(kA)_{11} + (kA)_{22} = k(0) = 0\]

因此,$kA$ 也满足条件,属于 $V$。

由此可知,$V$ 是 $R^{2\times2}$ 的子空间。

(2) 要求 $V$ 的维数和

一个基,我们可以先找到一个基,然后计算其维数。

我们已知 $V$ 的元素满足 $a_{11} + a_{22} = 0$。可以选择以下矩阵作为基:

\[ B_1 = \begin{bmatrix} 1 & 0 \\ 0 & -1 \end{bmatrix}, \quad B_2 = \begin{bmatrix} 0 & 1 \\ 0 & 0 \end{bmatrix}, \quad B_3 = \begin{bmatrix} 0 & 0 \\ 1 & 0 \end{bmatrix} \]

这三个矩阵线性无关,并且任意满足条件 $a_{11} + a_{22} = 0$。因此,$\{B_1, B_2, B_3\}$ 是 $V$ 的基。

由于 $\{B_1, B_2, B_3\}$ 是 $V$ 的基,所以 $\text{span}(\{B_1, B_2, B_3\}) = V$。我们需要确定基的个数。

观察矩阵 $B_1, B_2, B_3$,它们都是 $2 \times 2$ 的矩阵,因此,$\text{span}(\{B_1, B_2, B_3\})$ 的维数最多为 $2 \times 2 = 4$。

然而,我们可以注意到 $B_1, B_2, B_3$ 中存在一个线性关系:

\[ B_1 + B_2 + B_3 = \begin{bmatrix} 1 & 0 \\ 0 & -1 \end{bmatrix} + \begin{bmatrix} 0 & 1 \\ 0 & 0 \end{bmatrix} + \begin{bmatrix} 0 & 0 \\ 1 & 0 \end{bmatrix} = \begin{bmatrix} 1 & 1 \\ 1 & -1 \end{bmatrix} \]

由于 $1 + (-1) = 0$,我们可以发现 $B_1 + B_2 + B_3$ 也属于 $V$。因此,$\text{span}(\{B_1, B_2, B_3\})$ 中存在冗余向量。

通过观察,我们可以发现 $\{B_1, B_2, B_3\}$ 中的向量是线性无关的。因此,$\text{span}(\{B_1, B_2, B_3\})$ 的维数为 3。

综上所述,$V$ 的维数为 3



\textbf{13 QUE}
\bigskip

试证明所有二阶矩阵之集合形成的实线性空间是所有二阶实对称矩阵之集合形成的子空间与所有二阶反对称矩阵之集合形成的子空间的
直和

\textbf{13 ANS}
\bigskip

我们需要证明两个条件:

1. 实对称矩阵子空间与实反对称矩阵子空间的交集只包含零矩阵。

2. 任意二阶矩阵可以表示为一个实对称矩阵和一个实反对称矩阵的和。

首先,我们定义实对称矩阵子空间为集合$S$,其中包含所有二阶实对称矩阵。实反对称矩阵子空间定义为集合$A$,其中包含所有二阶实反对称矩阵。

证明第一个条件:实对称矩阵子空间与实反对称矩阵子空间的交集只包含零矩阵。

设矩阵$M$同时属于$S$和$A$,即$M\in S$且$M\in A$。

证明第二个条件:任意二阶矩阵可以表示为一个实对称矩阵和一个实反对称矩阵的和。

设任意二阶矩阵$P$,我们将证明存在一个实对称矩阵$Q$和一个实反对称矩阵$R$,使得$P = Q + R$。

令$Q = \frac{1}{2}(P + P^T)$,$R = \frac{1}{2}(P - P^T)$。

首先验证$Q$是否为实对称矩阵:
$(Q^T)^T = \left(\frac{1}{2}(P + P^T)\right)^T = \frac{1}{2}(P^T + (P^T)^T) = \frac{1}{2}(P^T + P) = Q$。
因此,$Q$为实对称矩阵。

接下来验证$R$是否为实反对称矩阵:
$(R^T)^T = \left(\frac{1}{2}(P - P^T)\right)^T = \frac{1}{2}((P^T)^T - P^T) = \frac{1}{2}(P - P^T) = -R$。
因此,$R$为实反对称矩阵。

最后,我们有:
$Q + R = \frac{1}{2}(P + P^T) + \frac{1}{2}(P - P^T) = P$

由此可见,任意二阶矩阵可以表示为一个实对称矩阵和一个实反对称矩阵的和,所有二阶矩阵之集合形成的实线性空间是所有二阶实对称矩阵之集合形成的子空间与所有二阶反对称矩阵之集合形成的子空间的
直和



%%%%%%%%%%%%%%%%%%%%
\item \textit{HOMEWORK 3 {P77-78 1, 2, 6, 7}}%%% 
%%%%%%%%%%%%%%%%%%%%

\textbf{1 QUE}
\bigskip

判断下列变换中哪些是线性变换

\begin{enumerate}
	\item 在$R^3$中,设$x = (\xi_1, \xi_2, \xi_3), Tx = ({\xi_1}^2,{\xi_1} + {\xi_2},\xi_3)$
    \item 在矩阵空间中$R^{n \times n}$中,$Tx = BXC$,这里$B,C$是固定矩阵
    \item 在线性空间中$P_n$中,$Tf(t) = f(t + 1)$
\end{enumerate}


\textbf{1 ANS}
\bigskip

\begin{enumerate}
	\item 不是线性变换。
	
	因为${T(2x) = (4\xi_1^2, 2\xi_1 + 2\xi_2, 2\xi_3)}$,但是${2T(x) = (2\xi_1^2, 2\xi_1+ 2\xi_2, 2\xi_3)}$。
    \item 是线性变换。
    \item 是线性变换。
\end{enumerate}


\textbf{2 QUE}
\bigskip

在$R^2$中,设 $x = (\xi_1, \xi_2)$. 证明$T_1x = (\xi_2. -\xi_1)$与$T_2x = (\xi_1,-\xi_2)$是 $R^2$的两个线性变换,并求$T_1 + T_2, T_1T_2$及$T_2T_1$ 


\textbf{2 ANS}
\bigskip

	设${k,l \in R, y = (\alpha_1, \alpha_2)\in R^2}$,$kx + ly = (k\xi_1 + l\alpha_1, k\xi_2 + l\alpha_2)$
	
	${T_1(kx + ly) = (k\xi_2 + l\alpha_2,-k\xi_1 - l\alpha_1) = kT_1(x) + lT_1(y)}$

	所以$T_1$是线性变换,同理$T_2$也是线性变换.

	所以${(T_1 + T_2)x = T_1(x) + T_2(x) = (\xi_2 + \xi_1, -\xi_1 - \xi_2)}$

	${(T_1T_2)x = T_1T_2(x) = (-\xi_2, -\xi_1)}$

	${(T_2T_1)x = T_2T_1(x) = (\xi_2, \xi_1)}$.


\textbf{6 QUE}
\bigskip

六个函数

    $x_1 = e^{at} \cos{b} t$, \quad $x_2 = e_{at}\sin{b}t$, \quad $x_3 = t e^{at}\cos{b}t$

    $x_4 = t e^{at} \sin{b} t$, \quad $x_5 = \frac{1}{2} t^2 e^{at} \cos{b} t$, \quad $x_6 =  \frac{1}{2} t^2 e^{at}\sin{b}t$

    的所有实系数线性组合构成实数域R上的一个六维线性空间$V^6 = L(x_1, x_2,x_3,x_4,x_5,x_6)$,求微分变换D在基$x_1,x_2, \cdots, x_6$下的矩阵

\textbf{6 ANS}
\bigskip

${D_{x_1}} = ae^{at} \cos{b}t - e^{at} b\sin{b} t = ax_1 - bx_2$

${D_{x_2}} = ae^{at} \sin{b}t + e^{at} b\cos{b} t = bx_1 + ax_2$

${D_{x_3}} = e^{at} \cos{b}t + tae^{at} b\cos{b} t - te^{at} b\sin{b} t = x_1 + ax_3 -bx_4$

${D_{x_4}} = e^{at} \sin{b}t - tae^{at} b\cos{b} t + te^{at} b\cos{b} t= x_2 + bx_3 + ax_4$

${D_{x_5}} = te^{at} \cos{b}t + \frac{1}{2}t^2ae^{at} b\sin{b} t - \frac{1}{2}t^2e^{at} b\sin{b} t= x_3 + ax_5 - bx_6$

${D_{x_6}} = te^{at} \sin{b}t + \frac{1}{2}t^2ae^{at} \sin{b} t + + \frac{1}{2}t^2e^{at} b\cos{b} t = x_4 + bx_5 + ax_6$

所以


\[D = \begin{bmatrix}
    a & b & 1 & 0 & 0 & 0 \\
	-b & a & 0 & 1 & 0 & 0 \\
	0 & 0 & a & b & 1 & 0 \\
	0 & 0 & -b & a & 0 & 1 \\
	0 & 0 & 0 & 0 & a & b \\
	0 & 0 & 0 & 0 & -b & a
\end{bmatrix}\]

\textbf{7 QUE}
\bigskip

已知$R^3$的线性变换T在基$x_1 = (-1,1,1), x_2 = (1,0,-1),x_3 = (0,1,1)$下的矩阵是
\[\begin{bmatrix}
    1 & 0 & 1 \\
	1 & 1 & 0 \\
	-1 & 2 & 1
\end{bmatrix}\]
求T在基${e_1 = (1, 0, 0), e_2 = (0, 1, 0), e_3 = (0, 0, 1)}$下的矩阵。

\textbf{7 ANS}
\bigskip

由题意得:
$C^{-1} = $
$\begin{bmatrix}
    1 & 0 & 1 \\
	1 & 1 & 0 \\
	-1 & 2 & 1
\end{bmatrix}$

所以在新基下的矩阵为
$C^{-1}
\begin{bmatrix}
    1 & 0 & 1 \\
	1 & 1 & 0 \\
	-1 & 2 & 1
\end{bmatrix}C = 
\begin{bmatrix}
    -1 & 1 & -2 \\
	2 & 2 & 0 \\
	3 & 0 & 2
\end{bmatrix}
$


%%%%%%%%%%%%%%%%%%%%
\item \textit{HOMEWORK 4 {P106-107} 1 (1)(2), 2, 4, 5, 10, 11}%%% 
%%%%%%%%%%%%%%%%%%%%

\textbf{1(1)(2) QUE}
\bigskip

	设${x = (\xi_1, \xi_2, \dots, \xi_n)}$, $y = (\eta_1, \eta_2, \dots, \eta_n)$是${R^n}$的任意两个向量, $A = (a_{ij})_{n*n}$
	
	是正定矩阵,令$(x, y) = xAy^T$,则

	(1)证明在该定义下$R^n$形成欧式空间

	(2)求$R^n$对于单位向量$e_1 = (1, 0, 0, \dots, 0), e_2 = (0, 1, 0, \dots, 0), \dots, e_n = (0, 0,0, \dots, 0, 1)$的度量矩阵

\textbf{1(1)(2) ANS}
\bigskip

(1) 要证明$R^n$形成欧式空间,需要满足以下条件:

(i) 零向量的存在:由于$x = (\xi_1, \xi_2, \dots, \xi_n)$是$R^n$的任意向量,令$x = 0$,则有$(0, y) = 0Ay^T = 0$,因此零向量存在。

(ii) 向量的加法:满足定义要求。

(iii) 数乘:数乘满足定义要求。

(iv) 内积的存在:定义内积为$(x, y) = xAy^T$,根据矩阵的性质,内积满足交换律和线性性质:内积满足定义要求。

综上所述,$R^n$在给定的内积定义下满足欧式空间的所有条件。



(2)由$(e_i,e_j) = e_iA{e_j}^t = a_{ij}$知,$r^n$中基$e_1,e_2, \dots, e_n$的度量矩阵为A。


\textbf{2 QUE}
\bigskip

	设$x_1, x_2, \dots, x_n$是实线性空间$v^n$的基,向量$x = \xi_1x_1 +\xi_2x_2+\dots+\xi_nx_n, y = \eta_1x_1+ \eta_2x_2+\dots
	+\xi_nx_n$对应于实数$(x, y) = \sum_{i = 1}^ni\xi_i\eta_i$,试问$V^n$是否是欧式空间。


\textbf{2 ANS}
\bigskip

要确定$V^n$是否是欧式空间,我们需要验证欧式空间的定义条件:

(i) 零向量的存在:零向量表示为$x = 0x_1 + 0x_2 + \dots + 0x_n$,对应的实数为$(x, y) = \sum_{i=1}^ni(0)(0) = 0$。因此,零向量存在。

(ii) 向量的加法:向量的加法满足定义要求。

(iii) 数乘:数乘满足定义要求。

(iv) 内积的存在:定义内积为$(x, y) = \sum_{i=1}^ni\xi_i\eta_i$,根据内积的定义,内积满足交换律和线性性质:
\begin{align*}
(x, y) &= \sum_{i=1}^ni\xi_i\eta_i \\
&= \sum_{i=1}^ni\eta_i\xi_i \\
&= (y, x)
\end{align*}
\begin{align*}
(x, ay) &= \sum_{i=1}^ni\xi_i(a\eta_i) \\
&= a\sum_{i=1}^ni\xi_i\eta_i \\
&= a(x, y)
\end{align*}
\begin{align*}
(x, y+z) &= \sum_{i=1}^ni\xi_i(\eta_i+\zeta_i) \\
&= \sum_{i=1}^ni\xi_i\eta_i + \sum_{i=1}^ni\xi_i\zeta_i \\
&= (x, y) + (x, z)
\end{align*}

因此,内积满足定义要求。

综上所述,$V^n$在给定的内积定义下满足欧式空间的所有条件。
	


\textbf{4 QUE}
\bigskip

	在$R^4$中,求一单位向量与$(1, 1, -1, 1), (1, -1, -1, 1)$及$(2,1,1,13)$均正交

\textbf{4 ANS}
\bigskip

设该向量为 $(x, y, z, w)$,我们可以得到以下方程组:
\[ \begin{aligned}
(1, 1, -1, 1) \cdot (x, y, z, w) &= 0 \\
(1, -1, -1, 1) \cdot (x, y, z, w) &= 0 \\
(2, 1, 1, 13) \cdot (x, y, z, w) &= 0
\end{aligned}\]

该齐次线性方程组的非零解为 $x = (4, 0, 1, -3)$, 然后进行单位化。

因此,单位向量为:
$(\frac{4}{\sqrt[]{26}}, 0, \frac{1}{\sqrt[]{26}}, \frac{-3}{\sqrt[]{26}})$。



%%%% 从这里开始错误

\textbf{5 QUE}
\bigskip

设$x_1, x_2, x_3, x_4, x_5$是欧式空间$V^5$的一个标准正交基。$V_1 = L(y_1, y_2, y_3)$,其中$y_1 = x_1 + x_5, y_2 = x_1 - x_2 + x_4, y_3 = 2x_1 + x_2 +x_3$,求
$V_1$的一个标准正交基

\textbf{5 ANS}
\bigskip


因为$\alpha_1, \alpha_2,\alpha_3$线性无关,所以$y_1, y_2, y_3$线性无关。

因此,$y_1, y_2, y_3$就是$V_1$的一个基,我们只需要对它们进行单位化即可得到一个标准正交基。


现在,我们可以计算标准正交基:

\[
u_1 = \frac{y_1}{\|y_1\|}, \quad u_2 = \frac{y_2}{\|y_2\|}, \quad u_3 = \frac{y_3}{\|y_3\|}
\]

其中,$u_1, u_2, u_3$是$V_1$的标准正交基。
请注意,上述计算中使用了向量的点乘和长度的定义。

\textbf{10 QUE}
\bigskip

设$T$是欧式空间$V$中的线性变换,且对$x, y \in V$,有
\[ (Tx, y) = -(x, Ty)\]
则称$T$为反对称变换,证明$T$为反对称变换的充要条件是,T在V的标准正交基下的矩阵$A$为反对称矩阵,即有$A^T = -A$

\textbf{10 ANS}
\bigskip


证明必要性:

假设$T$是反对称变换,我们需要证明$T$在$V$的标准正交基下的矩阵$A$满足$A^T = -A$。

由于我们考虑的是$V$的标准正交基,设该基为$\{v_1, v_2, \ldots, v_n\}$,则对于任意的$i$和$j$,有$(v_i, v_j) = \delta_{ij}$,其中$\delta_{ij}$是Kronecker delta符号,当$i=j$时为1,否则为0。

设向量$v$在基$\{v_1, v_2, \ldots, v_n\}$下的坐标表示为$[v]_\mathcal{B} = \begin{bmatrix} a_1 \\ a_2 \\ \vdots \\ a_n \end{bmatrix}$,其中$\mathcal{B}$是基$\{v_1, v_2, \ldots, v_n\}$。

考虑$(Tv, v)$的内积,根据反对称变换的定义有:
\[(Tv, v) = -(v, Tv)\]

将$v$的坐标表示和$T$的矩阵表示相结合,可以得到:
\[(Tv, v) = ([Tv]_\mathcal{B})^T [v]_\mathcal{B} = \begin{bmatrix} a_1 & a_2 & \cdots & a_n \end{bmatrix}^T A \begin{bmatrix} a_1 \\ a_2 \\ \vdots \\ a_n \end{bmatrix}\]

同样地,我们可以得到:
\[(v, Tv) = [v]_\mathcal{B}^T A^T [v]_\mathcal{B} = \begin{bmatrix} a_1 & a_2 & \cdots & a_n \end{bmatrix}^T A^T \begin{bmatrix} a_1 \\ a_2 \\ \vdots \\ a_n \end{bmatrix}\]

由于$(Tv, v) = -(v, Tv)$,我们有:
\[\begin{bmatrix} a_1 & a_2 & \cdots & a_n \end{bmatrix}^T A \begin{bmatrix} a_1 \\ a_2 \\ \vdots \\ a_n \end{bmatrix} = -\begin{bmatrix} a_1 & a_2 & \cdots & a_n \end{bmatrix}^T A^T \begin{bmatrix} a_1 \\ a_2 \\ \vdots \\ a_n \end{bmatrix}\]

对于任意的向量$v$,上式成立,因此我们可以得到:
\[A = -A^T\]

即矩阵$A$为反对称矩阵,证明了必要性。

充分性:

设向量$v$在基$\{v_1, v_2, \ldots, v_n\}$下的坐标表示为$[v]_\mathcal{B} = \begin{bmatrix} a_1 \\ a_2 \\ \vdots \\ a_n \end{bmatrix}$,其中$\mathcal{B}$是基$\{v_1, v_2, \ldots, v_n\}$。

根据$T$的矩阵表示,我们有$[Tv]_\mathcal{B} = A[v]_\mathcal{B}$。

考虑$(Tv, v)$的内积,根据矩阵乘法的定义有:
\[(Tv, v) = ([Tv]_\mathcal{B})^T [v]_\mathcal{B} = (A[v]_\mathcal{B})^T [v]_\mathcal{B}\]

展开上式,我们可以得到:
\[(Tv, v) = (A[v]_\mathcal{B})^T [v]_\mathcal{B} = ([v]_\mathcal{B})^T A^T [v]_\mathcal{B}\]

由于$A$为反对称矩阵,即$A^T = -A$,上式可以继续化简为:
\[(Tv, v) = ([v]_\mathcal{B})^T A^T [v]_\mathcal{B} = -([v]_\mathcal{B})^T A [v]_\mathcal{B}\]

而根据内积的定义,我们有$(v, Tv) = -([v]_\mathcal{B})^T A [v]_\mathcal{B}$。

由于$(Tv, v) = -(v, Tv)$,我们得到了反对称变换的定义。因此,$T$是反对称变换。

综上所述,$T$为反对称变换的必要条件是$T$在$V$的标准正交基下的矩阵$A$为反对称矩阵,即$A^T = -A$。

\textbf{11 QUE}
\bigskip

对于下列矩阵A,求正交(酉)矩阵P,使$P^{-1}AP$为对角矩阵:

\begin{enumerate}
	\item A = $\begin{bmatrix}
		2 & 2 & -2 \\
		2 & 5 & -4 \\
		-2 & -4 & 5  
		\end{bmatrix}$
	
	\item A = $\begin{bmatrix}
		0 & j & 1 \\
	-j & 0 & 0 \\
	1 & 0 & 0
	\end{bmatrix}$

\end{enumerate}




\textbf{11 ANS}
\bigskip

(1) 对于矩阵A = $\begin{bmatrix}
2 & 2 & -2 \\
2 & 5 & -4 \\
-2 & -4 & 5  
\end{bmatrix}$



(2) 对于矩阵A = $\begin{bmatrix}
0 & j & 1 \\
-j & 0 & 0 \\
1 & 0 & 0
\end{bmatrix}$


%%%%%%%%%%%%%%%%%%%%
\item \textit{HOMEWORK 5 {P79 19(1)(3)}}%%% 
%%%%%%%%%%%%%%%%%%%%


\textbf{19(1)(3) QUE}
\bigskip

求下列各矩阵的Jordan标准型。

(1) $\begin{bmatrix}
	1 & 2 & 0 \\
0 & 2 & 0 \\
-2 & -1 & -1  
\end{bmatrix}$

(3)$\begin{bmatrix}
	3 & 1 & 0 & 0 \\
-4 & -1 & 0 & 0\\
7 & 1 & 2 & 1 \\
-7 & -6 & -1 & 0  
\end{bmatrix}$

\textbf{19(1)(3) ANS}
\bigskip


(1)$det(\lambda I - A) = (\lambda - 1)(\lambda - 2)(\lambda + 1)$, A有三个不同的特征值,从而A的Jordan标准型
为$\begin{bmatrix}
	1 &  &  \\
 & 2 &  \\
 &  & -1  
\end{bmatrix}$

(3)$det(\lambda I - A) = (\lambda - 1)^4$,由此可知$D_4(\lambda) = (\lambda)^4$,同时可知 $D_1(\lambda) = 1$

容易求得$D_2(\lambda) = 1$,位于$\lambda - A$的第2, 3, 4行与第1, 2, 4列处的三阶子式 $ = 7 \lambda^2 - 4 \lambda + 17$
他和$D_4(\lambda)$互质,所以$D_3(\lambda) = 1$,于是$A$的Jordan标准型为

\[ j = \begin{bmatrix}
	1 & 1 & 0 & 0 \\
	0 & 1 & 1 & 0 \\
	0 & 0 & 1 & 1 \\
	0 & 0 & 0 & 1 \\
\end{bmatrix}\]


%%%%%%%%%%%%%%%%%%%%
\item \textit{HOMEWORK 6 {P163 3, 4, 5, 6}}%%% 
%%%%%%%%%%%%%%%%%%%%

\textbf{3 QUE}
\bigskip

若A为是实反对称矩阵$(A^T = -A)$, 则$e^A$为正交矩阵。

\textbf{3 ANS}
\bigskip

若证明为正交矩阵,只需证明$e^A {e^A}^T = I$.

$e^A({e^A}^T) = e^A {e^A}^T = e^{A - A} = I$  

所以$e^A$为正交矩阵。

\textbf{4 QUE}
\bigskip

若A是Hermite矩阵,则$e^{iA}$是酉矩阵。

\textbf{4 ANS}
\bigskip

若证明为正交酉矩阵,只需证明$e^{iA} {e^{iA}}^T = I$.

$e^{iA} {e^{iA}}^T = e^{iA} {e^{-iA}} = e^{O} = I$  

所以$e^{iA}$为正交酉矩阵。

\textbf{5 QUE}
\bigskip

设  $A = \begin{bmatrix}
	2 & 1 & 0 \\
	0 & 0 & 1 \\
	0 & 1 & 0
\end{bmatrix}$,求$e^A, e^{tA}(t \in R), \sin A$

\textbf{5 ANS}
\bigskip

$\det(\lambda I - A) =  0 = (\lambda -2)(\lambda + 1)(\lambda - 1)$

得到特征值为2 对应的特征向量为$(1, 0 ,0)$

得到特征值为1 对应的特征向量为$(-1, 1 ,1)$

得到特征值为-1 对应的特征向量为$(1, -3 ,3)$

即存在可逆矩阵$P = \begin{bmatrix}
	1 & -1 & 1 \\
	0 & 1 & -3 \\
	0 & 1 & 3
\end{bmatrix}$
使得$P^{-1}AP = \begin{bmatrix}
	2 & 0 & 0 \\
	0 & 1 & 0 \\
	0 & 0 & -1 
\end{bmatrix}$


同时$P^{-1} = \frac{1}{6} \begin{bmatrix}
	6 & 4 & 2 \\
	0 & 3 & 3 \\
	0 & -1 & 1
\end{bmatrix}$


因此,根据以上的分析:

(1)$e^A = Pdiag(e^2, e, e^{-1})P^{-1} = $

\[ \frac{1}{6}  \begin{bmatrix}
	6e^2 & 4e^2 - 3e + e^{-1} & 2e^2 - 3e +e^{-1} \\
	0 & 3e +3e^{-1} & 3e -3e^{-1} \\
	0 & 3e -3e^{-1} & 3e + 3e^{-1} 
\end{bmatrix}\]

(2)$e^{tA} = Pdiag(e^{2t}, e^t, e^{-t})P^{-1} = $

\[ \frac{1}{6}  \begin{bmatrix}
	6e^{2t} & 4e^{2t}-3e^t-e^{-t} & 2e^{2t} -3e^t + t^-t \\
	0 & 3e^t +3e^{-t} & 3e^t -3e^{-t} \\
	0 & -e^t + 3e^{-t} & 3e^t + 3e^{-t}
\end{bmatrix}\]

(3)$e^A = Pdiag(\sin(2), \sin(1), \sin(-1))P^{-1} = $

\[ \frac{1}{6}  \begin{bmatrix}
	6sin2 & 4sin2 -3sin1 -sin(-1) & 2sin2 -3sin1 + sin(-1) \\
	0 & 3sin1 + 3sin(-1) & 3sin1 -3sin(-1) \\
	0 & 3sin1 - 3sin(-1) & 3sin1 + 3sin(-1)
\end{bmatrix}\]

\textbf{6 QUE}
\bigskip

设$f(z) = \ln(z)$, 求$f(A)$,这里A为

(1) $\begin{bmatrix}
	1 & 0 & 0 & 0 \\
	1 & 1 & 0 & 0 \\
	0 & 1 & 1 & 0 \\
	0 & 0 & 1 & 1
\end{bmatrix}$
\qquad \qquad (2)$\begin{bmatrix}
	2 & 1 & 0 & 0 \\
	0 & 2 & 0 & 0 \\
	0 & 0 & 1 & 1 \\
	0 & 0 & 0 & 1
\end{bmatrix}$

\textbf{6 ANS}
\bigskip

(1)因为反斜对角单位矩阵有特殊的用法。 因此$P = \begin{bmatrix}
	 & & & 1 \\
	 & & 1 & \\
	 & 1 & & \\
	1 & & &
\end{bmatrix}$,
$P^{-1}AP = J = \begin{bmatrix}
	1 & 1 & & \\
	& 1 & 1 & \\
	& & 1 & 1 \\
	& & & 1 
\end{bmatrix}$

所以$\ln(A) = P \ln(J) p^{-1} = \begin{bmatrix}
	0 & 0 & 0 & 0 \\
	1 & 0 & 0 & 0 \\
	-\frac{1}{2} & 1 & 0 & 0 \\
	\frac{1}{3} & -\frac{1}{2} & 1 & 0
\end{bmatrix}$


(2)$A = \begin{bmatrix}
	J_1 & \\
	& J_2 
\end{bmatrix}$

$J_1 = \begin{bmatrix}
	2 & 1\\
	0 & 2 
\end{bmatrix}$,$J_2 = \begin{bmatrix}
	1 & 1\\
	0 & 1 
\end{bmatrix}$

$\ln(A) = \begin{bmatrix}
	\ln(j_1) & \\
	& \ln(J_2)
\end{bmatrix} = \begin{bmatrix}
	\ln(2) & \frac{1}{2} & 0 & 0 \\
	0 & \ln(2) & 0 & 0 \\
	0 & 0 & 0 & 1 \\
	0 & 0 & 0 & 0
\end{bmatrix}$



%%%%%%%%%%%%%%%%%%%%
\item \textit{HOMEWORK 8 {P170-171 5, 9 P177 3, 4}}%%% 
%%%%%%%%%%%%%%%%%%%%

\textbf{5 QUE}
\bigskip


\textbf{5 ANS}
\bigskip

\textbf{9 QUE}
\bigskip

\textbf{9 ANS}
\bigskip

\textbf{3 ANS}
\bigskip

\textbf{3 ANS}
\bigskip

\textbf{4 ANS}
\bigskip

\textbf{4 ANS}
\bigskip

%%%%%%%%%%%%%%%%%%%%
\item \textit{HOMEWORK 9 {P195 2, 3}}%%% 
%%%%%%%%%%%%%%%%%%%%

\textbf{2 ANS}
\bigskip

证明式(4.1.30).

\textbf{2 ANS}
\bigskip


\textbf{3 ANS}
\bigskip

设A为实对称正定矩阵,且Gauss消去法第一步得到的矩阵为 \[ 
	\]
证明B仍是实对称正定矩阵,且对角元素不增加。

\textbf{3 ANS}
\bigskip



%%%%%%%%%%%%%%%%%%%%
\item \textit{HOMEWORK 10 {P219-220 1, 7, 8}}%%% 
%%%%%%%%%%%%%%%%%%%%

\textbf{1 QUE}
\bigskip

用Schmidt正交化方法求矩阵 \[ A = \begin{bmatrix}
	0 & 1 & 1 \\
	1 & 1 & 0 \\
	1 & 0 & 1 
\end{bmatrix}\]的QR分解。

\textbf{1 ANS}
\bigskip

\textbf{7 QUE}
\bigskip

用Givens变换求矩阵
\[ A = \begin{bmatrix}
	2 & 2 & 1 \\
	0 & 2 & 2\\
	2 & 1 & 2
\end{bmatrix}\]的QR分解

\textbf{7 ANS}
\bigskip

\textbf{8 QUE}
\bigskip

用Householder求变换矩阵 \[ A = \begin{bmatrix}
	0 & 4 & 1 \\
	1 & 1 & 1 \\
	0 & 3 & 2
\end{bmatrix}\]的QR分解


\textbf{8 ANS}
\bigskip

%%%%%%%%%%%%%%%%%%%%
\item \textit{HOMEWORK 11 {P225 1(2), 2, 5 P233 1}}%%% 
%%%%%%%%%%%%%%%%%%%%

\textbf{1(2) QUE}
\bigskip

求下列各矩阵的满秩分解。

(2) $ \begin{bmatrix}
	1 & -1 & 1 & 1 \\
	-1 & 1 & -1 & -1 \\
	-1 & -1 & 1 & 1 \\
	1 & 1 & -1 & -1 
\end{bmatrix}$

\textbf{2 QUE}
\bigskip

设$B = \in R_r^{m \times r}(r > 0)$,证明$B^TB$非奇异。

\textbf{2 ANS}
\bigskip

\textbf{5 QUE}
\bigskip

证明 \[ rankA = rank(A^TA) = rank(AA^T)\]
这里$A \in R_r^{m \times r}$.

\textbf{5 ANS}
\bigskip

\textbf{1 QUE}
\bigskip

设 $\sigma_1$ 和 $\sigma_n$ 是矩阵A的最大奇异值和最小奇异值。证明: $\sigma_1 = {\parallel A \parallel}_2$;
当A是非奇异矩阵时,${\parallel A^{-1} \parallel}_2 = \frac{1}{\sigma_n}$


\textbf{1 ANS}
\bigskip



%%%%%%%%%%%%%%%%%%%%
\item \textit{HOMEWORK 12 {P306 3, 4, 5}}%%% 
%%%%%%%%%%%%%%%%%%%%



%%%%%%%%%%%%%%%%%%%%
\item \textit{HOMEWORK 13 {P306-307 6, 8, 11, 12}}%%% 
%%%%%%%%%%%%%%%%%%%%

%%%%%%%%%%%%%%%%%%%%
\item \textit{HOMEWORK 14 {P295 1, 4}}%%% 
%%%%%%%%%%%%%%%%%%%%


%%%%%%%%%%%%%%%%%%%%
\item \textit{HOMEWORK 15 {P332 2 3(1)(2)}}%%% 
%%%%%%%%%%%%%%%%%%%%

%%%%%%%%%%%%%%%%%%%%
\item \textit{HOMEWORK 16 {P343-344 1, 2, 5}}%%% 
%%%%%%%%%%%%%%%%%%%%


%%%%%%%%%%%%%%%%%%%%
\item \textit{HOMEWORK 17 {P275 1, 2}}%%% 
%%%%%%%%%%%%%%%%%%%%

%%%%%%%%%%%%%%%%%%%%
\item \textit{HOMEWORK 18 {P261 2, 3, 4}}%%% 
%%%%%%%%%%%%%%%%%%%%

\end{enumerate}





\end{document}

